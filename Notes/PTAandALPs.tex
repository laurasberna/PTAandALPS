\documentclass[preprintnumbers,nofootinbib,prd,superscriptaddress,groupedaddress,aps]{revtex4-1} %twocolumn,

%\pdfoutput=1

%\documentclass[nofootinbib,aps,showpacs,floats,letterpaper,floatfix,groupedaddress,twocolumn]{revtex4}
%\documentclass[twocolumn,showpacs,preprintnumbers,amssymb]{revtex4}
%\documentclass[preprint,showpacs,preprintnumbers,amsmath,amssymb,nofootinbib]{revtex4}
%\documentclass[showpacs,preprintnumbers,amsmath,amssymb,nofootinbib]{revtex4}

%\usepackage[utf8]{inputenc}

\usepackage{graphicx,amssymb,amsmath,amsthm,amsfonts,epsfig,epsf}
\usepackage[outdir=./]{epstopdf}
\usepackage[linktocpage]{hyperref}
\usepackage[usenames]{color}
\usepackage{epstopdf}

\usepackage{aas_macros}
\usepackage{bm}
\usepackage{dcolumn}
\usepackage[latin1]{inputenc}
\usepackage{latexsym}
\usepackage{rotating}
% \usepackage{hyperref}
\usepackage{color}
\usepackage{longtable}
\usepackage{enumerate}
\usepackage{tensor}
\usepackage{stmaryrd}
\usepackage{verbatim}

%\usepackage{mathtools}
\usepackage{url}
\setlength{\tabcolsep}{12pt}

\def\nn{\nonumber}
\def\pa{\partial}
\def\be{\begin{equation}}
\def\ee{\end{equation}}
\def\beq{\begin{eqnarray}}
\def\eeq{\end{eqnarray}}
%%%%%%
\def\lp{{\ell+1}}
\def\lm{{\ell-1}}
\def\lpp{{\ell+2}}
\def\lmm{{\ell-2}}
%%%%%
\def\k{\kappa}
\def\d{\delta}
\def\s{\sigma}
\def\o{\omega}
\def\ob{\bar{\omega}}
\def\ph{\varphi}
\def\th{\vartheta}
\def\ii{{{\rm i}\,}}
\def\zz{\left<0\right>}
\def\cQ{{\cal Q}}
\def\el{{e^{\lambda}}}
\def\eml{{e^{-\lambda}}}
\def\bg{{\bar\gamma}}

\DeclareMathOperator\arctanh{tanh^{-1}}
\newcommand{\ac}[1]{{\textcolor{blue}{\sf{[AC: #1]}} }}
\newcommand{\ls}[1]{{\textcolor{red}{\sf{[LS: #1]}} }}

\def\mf{\mathfrak}
\newcommand{\PBH}{{\mbox{\tiny PBH}}}
\newcommand{\DM}{{\mbox{\tiny DM}}}
% \setcounter{tocdepth}{0}

\begin{document}

\title{Probing axion-like particles and dilatons with pulsar timing}

%%%%
%\author{Laura Sberna}\email{lsberna@perimeterinstitute.ca}
%\affiliation{Perimeter Institute for Theoretical Physics, 31 Caroline Street North Waterloo, Ontario N2L 2Y5, Canada.}
%
%\author{Paolo Pani}\email{paolo.pani@roma1.infn.it}
%\affiliation{Dipartimento di Fisica, ``Sapienza'' Universit\`a di Roma \& Sezione INFN Roma1, Piazzale Aldo Moro 5, 00185, Roma, Italy}
% \affiliation{CENTRA, Departamento de F\'{\i}sica, Instituto Superior T\'ecnico -- IST, Universidade de Lisboa -- UL, Avenida Rovisco Pais 1, 1049 Lisboa, Portugal}
%%%%
%\maketitle

\section{Effects in the magnetosphere}
The equation for the EM modes in a plasma coupled to a dilaton field is
\begin{widetext}
	\begin{flushleft}
\begin{align}\label{eq:modesgeneral}
\frac{\left(n^2-1\right) \left(\omega_p^2 \left(n^2 \cos ^2(\theta )-1\right)-\left(n^2-1\right) \omega ^2 \gamma _c^3 (n \,v \cos (\theta )-1)^2\right)}{\omega ^2 \gamma _c^3 (n \, v \cos (\theta )-1)^2}%+\nonumber \\
-g^2B^2\frac{ \left(n^2-1\right) \left( n^2 \cos (2 \theta )+n^2-2 \right)}{2 \left(m^2+\left(n^2-1\right) \omega ^2\right)}=0
\end{align}
\end{flushleft}
\end{widetext}
The interior of the magnetosphere, at $ r\ll r_A $ has
\begin{equation}\label{key}
A_p=\frac{\omega_p^2 \gamma_c}{\omega^2}\gg 1 \, ,
\end{equation}
while the exterior, at $ r\gg r_A $ has
\begin{equation}\label{key}
A_p \ll 1 \, .
\end{equation}
Among the modes, solutions of Eq.~\eqref{eq:modesgeneral}, one must chose the ones that propagate as transverse at larger angles. The trivial mode that is always transverse and always a solution is
\begin{equation}\label{key}
n_1=1 \, .
\end{equation}
Since we already know this solution, we can divide the Eq.~\eqref{eq:modesgeneral} by $ (n^2 -1) $ and look for the other ones. 
We work in the small angles approximation $ \theta \ll 1 $. Moreover, the velocity of the plasma is close to $ c=1 $, $ \gamma_c\gg 1 $ and $ v \approx 1- \frac{1}{2 \gamma_c} $.
 
\subsection{Magnetosphere interior}
In the magnetosphere exterior, where $ A_p \gg 1 $ we can take $ v=1 $. We pick the mode that becomes transverse when $ \theta \gg \theta_*= (\frac{\omega_p^2 \gamma_c^-3}{\omega^2})^{1/4}$ \ls{there's a typo in the definition of this quantity in the draft},
\begin{equation}\label{key}
n_2=1+\frac{\theta ^2}{4}-\sqrt{\theta ^4+\theta _*^4} + F(g,\omega,\theta,\theta_*,m) .
\end{equation}
In the expression above, the function $ F $ is zero in the absence of the coupling $ g $, and was introduced to find the coupled solution perturbatively. At first order in $ F $ and $ g^2 $, we obtain \ls{different from the draft}
\begin{widetext}
	\begin{flushleft}
\begin{equation}\label{key}
F(g,\omega,\theta,\theta_*,m)=\frac{B^2 g^2 \left(G\left(\theta ,\theta _*\right)+\theta ^2\right){}^2 \left(\theta ^2 G\left(\theta ,\theta _*\right)+\theta ^4+8 \theta _*^4\right)}{2 \left(\left(G\left(\theta ,\theta _*\right)+\theta ^2\right){}^2+16 \theta _*^4\right) \left(\theta _*^4 \omega ^2 \left(G\left(\theta ,\theta _*\right)-\theta ^2-8\right)+m^2 \left(G\left(\theta ,\theta _*\right)+\theta ^2\right)\right)}
\end{equation}
\end{flushleft}
\end{widetext}

\subsection{Magnetosphere exterior}
In the magnetosphere exterior, where $ A_p \ll 1 $ and $ |1-n|\ll \gamma^{-2} $, the transverse modes at zero coupling $ g=0 $ are $ n_1=1 $ and \ls{different from the draft}
\begin{equation}\label{key}
n_2=1-2 A_p \theta ^2 + O(\theta^4, A_p^2)\ .
\end{equation}
With the same approximations, we arrive, for $ g\neq 0 $, at
\begin{equation}\label{key}
n_2= 1+ \frac{B^2 g^2 }{2 \left(B^2 g^2+m^2\right)} \theta ^2 + + O(\theta^4, A_p)
\end{equation}
The new mode is bigger than the $ g=0 $ contribution, which was therefore discarded \ls{Are we so sure about this?}.

% \bibliographystyle{apsrev4}
\bibliographystyle{utphys}
\bibliography{Ref}
% 
\end{document}

%%%%%%%%%%%%%%%%%%%%%%%%%%%%%%%%%%%%%%%%
